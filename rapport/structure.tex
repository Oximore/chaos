
Les threads peuvent être stockés dans différentes structures. En effet, on peut facilement passer d'une structure à l'autre en changeant une option de compilation. Cela se fait de la même manière que pour effectuer le choix entre pthread et la librairie implémentée.
Ceci nous a permis de tester facilement l'efficacité de ces différentes structures.

\subsection{Structure File}
\subsubsection{Type de structure}
Il s'agit d'une liste d'élément chaînés implémentée comme une FIFO. Nous avons choisi de mettre un pointeur sur le dernier élément de la liste pour des raisons pratiques. Les threads sont encapsulés dans des structures element.
\begin{verbatim}
struct list 
{
  struct element * first;
  struct element * last;
};

struct element
{
  struct element * next;
  thread_t thread;
};

\end{verbatim}


\subsubsection{Fonctions utilisées}
Au lieu d'implémenter directement une gestion des priorités, les threads sont sélectionnés en tête (afin d'être exécutés), et ajoutés en queue. On utilise donc un structure de donnée très simple.

\subsection{Structure Arbre}
\subsubsection{Type de structure}
Nous avons décidé de créer une structure de type arbre binaire de recherche équilibré pour faciliter l'implémentation éventuelle des priorités.

En effet, l'arbre est trié selon les priorités, cela apporte une amélioration en terme de complexité pour la recherche de thread (dans le cadre d'une implémentation des priorités).

\begin{verbatim}
struct tree {
  struct node * root;
};

struct node {
  struct node * left;
  struct node * right;
  struct node * root;
  thread_t thread; 
};

\end{verbatim}

Chaque structure node possède les champs left et right qui correspondent aux fils gauche et droit.
Le champ root correspond au père du noeud. Ce champ est nécessaire afin de pouvoir équilibrer l'arbre après chaque insertion et suppression de noeud.

\subsubsection{Fonctions utilisées}
Les threads sont retirés tout à gauche de l'arbre (car ils possède la priorité la plus basse).

L'ajout de thread se fait le plus à droite possible dans l'arbre (c'est à dire que s'il a la même priorité qu'un thread de l'arbre, il sera placé à sa droite, et donc exécuté après)
La fonction d'équilibrage se charge (grâce à des rotations droite et gauche) d'équilibrer tous les ancêtres du noeud ajouté ou supprimé.
