\section{Avancement actuel}
À l'heure actuelle nous avons implémenté la totalité des fonctions de l'interface \textit{thread.h} proposée. 

\subsection{thread\_self}
Cette fonction retourne juste le pointeur vers notre structure.
L'adresse en mémoire de celle ci étant unique, celà fait un identifiant unique pour chaque thread (une fois casté en int).

\subsection{thread\_create}
Cette fonction créer un nouveau thread en allouant la place nécessaire à un \textit{struct thread} ainsi qu'à une pile de la même taille que le programme courant. On remplit le \textit{u\_context} avec le context courant puis on appelle \textit{makecontext()} avec une fonction qui se chargera d'apeller la fonction passé en paramètre puis stockera son résultat et enfin détruira le thread en désalouant la mémoire consomée par le \textit{u\_context}.

\subsection{thread\_yield}
Elle remplace le thread courrant avec celui qui est retirer de la pile avec \textit{get\_lower\_priority\_thread()}. Dans le cas ou la pile est vide, $i.e.$ il n'y a pas d'autre thread que le thread courant, cette fonction ne fait rien.

\subsection{thread\_join}
Cette fonction se contente de vérifier si le thread passé en argument est terminé, si c'est le cas on récupère la valeur de retour stocker avant la destruction du thread. 


\subsection{thread\_exit}
Cette fonction termine et désaloue le thread courant.

\section{Travail à venir}
Cet ensemble de fonction nous servira à manipuler les threads. Nous allons coder des programmes de test pour simuler le fonctionnement de ces threads au fur et à mesure des différents changements de contexte.
