% thread.tex
\subsection{Structure thread}
Voici l'implémentation de la structure d'un thread ainsi que ces
différents attributs.

\subsubsection{Structure}

\begin{verbatim}
struct thread { 
 int priority;
 ucontext_t* context; 
 void * retval;
 struct thread* joiner; 
 int isfinished; 
};
\end{verbatim}

\subsubsection{priority}
Entier fixant la priorité d'exécution du thread. Pour le moment il
n'est pas utilisé.En effet, la gestion des priorités n'a pas été
implémentée.  Il devra être géré par les fonctions internes de la
structure \textit{tree}.

\subsubsection{context}
Structure de la librairie \textit{\textless
  ucontext.h\textgreater}. Sert à sauver notre contexte.
 
\subsubsection{retval}
Pointeur vers la valeur de retour du thread. Vaut $NULL$ tant qu'elle
n'a pas été remplie. Ce pointeur sert à sauvegarder la valeur de
retour du thread lorsque celui-ci termine.

\subsubsection{joiner}
Si un autre thread attend celui ci à l'aide de la fonction
\textit{thread\_join} alors il sera pointé par cette variable. On peut
ainsi facilement enlever le thread \textit{joiner} de la structure
(cela est plus efficace que de lui faire passer la main, car on ne
l'exécute plus). Quand le thread sur lequel le \textit{join} a été
fait se finit, le thread \textit{joiner} est à nouveau exécuté, puis
remis dans la structure.

\subsubsection{isfinished}
Cet entier peut prendre la valeur $0$ dans le cas où le thread n'a pas
fini son exécution, $1$ dans le cas contraire. Un thread fini n'est
pas remis dans la structure.


\subsection{Variables globales}
Pour pouvoir implémenter notre librairie nous avons fait le choix de
définir deux variables globales.
\begin{verbatim}
structure   thread_set;
struct thread* thread_current;
\end{verbatim}
\textit{thread\_set} correspond à une structure de donnée que nous
détaillerons plus tard. Il faut savoir que c'est la que sont stockés
les threads qui ne sont pas en cours.  \textit{thread\_current} est,
comme son nom l'indique, la structure qui contient les informations
sur le thread courrant.

\subsection{Fonction de la librairie}
Nous avons implémenté les fonctions dont le prototype est fourni dans
le fichier \textit{thread.h} du sujet. Voici leur différentes manières
de fonctionner.

\subsubsection{thread\_self()}
Cette fonction renvoie l'identifiant de thread. Ce dernier est défini
par l'adresse dans la pile de la structure thread correspondante.
% Celle ci est unique puisque à une adresse dans la pile ne peut
% correspondre qu'un même objet
Cas particulier de fonctionnement : si c'est la première fois que l'on
appelle une fonction de la librairie alors une structure thread est
crée pour le thread principal et l'objet \textit{thread\_set} est
initialisé.

\subsubsection{thread\_create()}
Cette fonction alloue et initialise une structure thread et son
contexte à l'aide de la fonction auxiliaire interne
\textit{thread\_init}.  Cette structure est ensuite stockée dans la
variable \textit{thread\_set}.

De même que la fonction précédente si c'est le
premier appel à une fonction de la librairie.



\subsubsection{thread\_yield()}
Cette fonction passe la main à un autre thread si possible. Elle
extrait un thread de \textit{thread\_set} grâce à la fonction
\textit{structure\_get()}. Un échange de contexte est
ensuite fait pour que le thread courrant ce retrouve dans
\textit{thread\_set} et que le thread retiré soit exécuté.

Si la valeur de retour de \textit{structure\_get()} est nulle, alors
soit \textit{thread\_set} n'a jamais été initialisé ($i.e.$ c'est le
premier appel à une fonction de la lib), soit il n'y a pas d'autre
thread en attente d'exécution. Dans les deux cas aucun changement de
context n'est fait.

\subsubsection{thread\_join()}
Cette fonction met en pause le thread courant en attendant la
terminaison du thread correspondant à l'id en paramètre.

Si le thread demandé n'est pas déjà fini alors on se met en attente
passive en faisant pointer son attribut \textit{joiner} vers le thread
courant avant de changer de thread et de stocker l'ancien dans
\textit{thread\_set}.

Dans le cas où le thread attendu est déjà fini, ou bien après
l'attente, on récupère la valeur de retour si le second argument n'est
pas null, puis on désalloue la structure du thread attendu, grace à la
fonction interne \textit{thread\_delete()}, puisque celui ci est fini.


\subsubsection{thread\_exit()}
Cette fonction interrompt le thread courrant.

On fixe la valeur de \textit{isfinished} à $1$, puis on change le
contexte. Pour cela, on recherche un nouveau thread à exécuter avec la
fonction \textit{structure\_get} ou on sélectionne un
thread ayant fait un \textit{thread\_join()} sur le thread actuel si
il en existe.

Dans le cas où la valeur de retour de \textit{structure\_get} est
nulle et qu'il n'y a pas de thread ayant fait de \textit{join} ou bien
que c'est le premier appel à une fonction de la librairie thread alors
on quitte le programme normalement avec un appel à \textit{exit(0)}.


\subsection{\textit{mthread}}
Nous avons implémenté une surcouche à notre librairie pour pouvoir
facilement l'interchanger avec la librairie \textit{\textless
  pthread.h\textgreater} de la librairie standard.

Cette surcouche, \textit{mthread\_t}, est implémentée dans
\textit{thread\_mix.c}.  Elle permet de faire appel aux fonctions
définies précédemment en remplaçant \textit{thread\_*} par
\textit{mthread\_*}. En compilant normalement, nos fonctions sont
appelées, avec la macro pré-processeur \textit{\# define
  MODE\_PTHREAD} ce sont les fonctions de la librairie standard qui
sont appelées.

Grâce à cette surcouche nous avons pu comparer les deux
implémentations facilement.

