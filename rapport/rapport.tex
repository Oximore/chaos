\documentclass[a4paper,11pt]{report}

\usepackage[french]{babel}
\usepackage[utf8]{inputenc}
\usepackage{graphicx}
\usepackage[colorlinks=true]{hyperref}
\hypersetup{urlcolor=blue,linkcolor=black,citecolor=black,colorlinks=true} 
%%%%%%%%%%%%%%%% Lengths %%%%%%%%%%%%%%%%
\setlength{\textwidth}{15.5cm}
\setlength{\evensidemargin}{0.5cm}
\setlength{\oddsidemargin}{0.5cm}

%%%%%%%%%%%%%%%% Variables %%%%%%%%%%%%%%%%

\title{Projet Système : Gestion de threads}
\author{Thibaud Cheippe, Lux Benjamin, Boutin Guillaume, Papa Gueye}


\begin{document}
\maketitle



%%%%%%%%%%%%%%%% Main part %%%%%%%%%%%%%%%%
\chapter*{Introduction}

Le but de ce projet est de coder en C une bibliothèque de threads en espace utilisateur. Nous avons commencé par nous familiariser avec les mécanismes de changement de contexte en exécutant le programme fourni en exemple, et en le modifiant. 

\chapter{Base de donnée} 

Grace à notre implémentation les threads de la runqueue peuvent être
stockés dans différentes structures. En effet, on peut facilement
passer d'une structure à l'autre en changeant une option de
compilation. Par défaut la structure d'arbre est choisie, pour
sélectioner la structure liste il suffit de compiler avec l'option
MODE\_LIST.Ceci nous a permis de tester facilement l'efficacité de ces
différentes structures.

\subsection{Structure list}
\subsubsection{Type de structure}
Il s'agit d'une liste d'élément chaînés implémentée comme une
FIFO. Nous avons choisi de mettre un pointeur sur le dernier élément
de la liste pour des raisons pratiques. Les threads sont encapsulés
dans des structures element.
\begin{verbatim}
struct list {
  struct element * first;
  struct element * last;
};

struct element {
  struct element * next;
  thread_t thread;
};

\end{verbatim}


\subsubsection{Fonctions utilisées}
Au lieu d'implémenter directement une gestion des priorités, les
threads sont sélectionnés en tête (afin d'être exécutés), et ajoutés
en queue. On utilise donc une structure de donnée très simple.  Les
fonctions d'ajout et de selection de threads se font en temps
constant.

\subsection{Structure Arbre}
\subsubsection{Type de structure}
Nous avons décidé de créer une structure de type arbre binaire de
recherche équilibré pour faciliter et optimisé l'implémentation
éventuelle des priorités.

En effet, l'arbre est trié selon les priorités, cela apporte une
amélioration en terme de complexité pour la recherche de thread (dans
le cadre d'une implémentation des priorités).

\begin{verbatim}
struct tree {
  struct node * root;
};

struct node {
  struct node * left;
  struct node * right;
  struct node * root;
  thread_t thread; 
};

\end{verbatim}

Chaque structure node possède les champs left et right qui
correspondent aux fils gauche et droit.  Le champ root correspond au
père du noeud. Ce champ est nécessaire afin de pouvoir équilibrer
l'arbre avec une faible complexité après chaque insertion et
suppression de noeud.

\subsubsection{Fonctions utilisées}
Les threads sont retirés tout à gauche de l'arbre (car ils possèdent
la priorité la plus basse).

L'ajout de thread se fait le plus à droite possible dans l'arbre
(c'est à dire que s'il a la même priorité qu'un thread de l'arbre, il
sera placé à sa droite, et donc exécuté après) La fonction
d'équilibrage se charge (grâce à des rotations droite et gauche)
d'équilibrer tous les ancêtres du noeud ajouté ou supprimé.

L'opération de selection est en $log(n)$ et celle de d'ajout est en
$nlog(n)$.


\chapter{Fonctions sur les Threads}
\section{Avancement actuel}
À l'heure actuelle nous avons implémenté la totalité des fonctions de l'interface \textit{thread.h} proposée. 

\subsection{thread\_self}
Cette fonction retourne juste le pointeur vers notre structure.
L'adresse en mémoire de celle ci étant unique, celà fait un identifiant unique pour chaque thread (une fois casté en int).

\subsection{thread\_create}
Cette fonction créer un nouveau thread en allouant la place nécessaire à un \textit{struct thread} ainsi qu'à une pile de la même taille que le programme courant. On remplit le \textit{u\_context} avec le context courant puis on appelle \textit{makecontext()} avec une fonction qui se chargera d'apeller la fonction passé en paramètre puis stockera son résultat et enfin détruira le thread en désalouant la mémoire consomée par le \textit{u\_context}.

\subsection{thread\_yield}
Elle remplace le thread courrant avec celui qui est retirer de la pile avec \textit{get\_lower\_priority\_thread()}. Dans le cas ou la pile est vide, $i.e.$ il n'y a pas d'autre thread que le thread courant, cette fonction ne fait rien.

\subsection{thread\_join}
Cette fonction se contente de vérifier si le thread passé en argument est terminé, si c'est le cas on récupère la valeur de retour stocker avant la destruction du thread. 


\subsection{thread\_exit}
Cette fonction termine et désaloue le thread courant.

\section{Travail à venir}



\end{document}
