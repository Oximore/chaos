\documentclass[a4paper,11pt]{report}

\usepackage[french]{babel}
\usepackage[utf8]{inputenc}
\usepackage{graphicx}
\usepackage[colorlinks=true]{hyperref}
\hypersetup{urlcolor=blue,linkcolor=black,citecolor=black,colorlinks=true} 
%%%%%%%%%%%%%%%% Lengths %%%%%%%%%%%%%%%%
\setlength{\textwidth}{15.5cm}
\setlength{\evensidemargin}{0.5cm}
\setlength{\oddsidemargin}{0.5cm}

%%%%%%%%%%%%%%%% Variables %%%%%%%%%%%%%%%%

\title{Projet Système : Gestion de threads}
\author{Thibaud Cheippe, Lux Benjamin, Boutin Guillaume, Papa Gueye}


\begin{document}
\maketitle



%%%%%%%%%%%%%%%% Main part %%%%%%%%%%%%%%%%
\section*{Introduction}

Le but de ce projet est de coder en C une bibliothèque de threads en espace utilisateur. Nous avons commencé par nous familiariser avec les mécanismes de changement de contexte en exécutant le programme fourni en exemple, puis nous l'avons modifié pour mieux comprendre le comportement des threads lors des changements de contexte. Nous avons implémenté une structure de données pour gérer les threads, et avons codé les premières fonctions de manipulation de threads.

\section*{Base de donnée} 
\section{}


\section*{Fonctions sur les Threads}
% thread.tex
\subsection{Structure thread}
\subsubsection{Structure}

\begin{verbatim}
struct thread{
  int priority;
  ucontext_t* context;
  void * retval;
  struct thread* joiner;
  int isfinished;
};
\end{verbatim}

\subsubsection{priority}
Entier fixant la prioritée d'exécution du thread. Pour le moment il
n'est pas utilisé.  Il devra être géré par les fonctions internes de
la structure \textit{data}.

\subsubsection{context}
Structure de la librairie \textit{\textless
ucontext.h\textgreater}. Sert à sauver notre contexte.
 
\subsubsection{retval}
Pointeur vers la valeur de retour du thread. Vaut $NULL$ tant qu'elle
n'a pas été remplie.

\subsubsection{joiner}
Si on autre thread attend celui ci à l'aide de la fonction
\textit{thread\_join} alors il sera pointé par cette variable.

\subsubsection{isfinished}
Cet entier peut prendre la valeur $0$ dans le cas où le thread n'a pas
fini son éxécution, $1$ dans le cas contraire.


\subsection{Variables globales}
Pour pouvoir implémenter notre librairie nous avons fait le choix de
définir deux variables globales.
\begin{verbatim}
struct data*   thread_data;
struct thread* thread_current;
\end{verbatim}
\textit{thread\_data} correspond à une structure de donnée que nous
détaillerons plus tard. Il faut savoir que c'est la que sont stocké
les threads qui ne sont pas en cours.  \textit{thread\_current} est,
comme son nom l'indique, la structure qui contient les informations
sur le thread courrant.

\subsection{Fonction de la librairie}
Nous avons implémenté les fonctions dont le prototype est fournit dans
le fichier \textit{thread.h} du sujet. Voici leur différente manière
de fonctioner.

\subsubsection{thread\_self()}
Cette fonction renvoie l'identifiant de thread. Ce dernier est définit
par l'adresse dans la pile de la structure thread correspondante.
% Celle ci est unique puisque à une adresse dans la pile ne peut
% correspondre qu'un même objet
Cas particulier de fonctionnement : si c'est la première fois que l'on
appelle une fonction de la librairie alors une structure thread est
crée pour le thread principal et l'objet \textit{thread\_data} est
initialisé.

\subsubsection{thread\_create()}
Cette fonction aloue et initialise une structure thread et son
contexte à l'aide de la fonction auxiliaire interne
\textit{thread\_init}.  Cette structure est ensuite stocker dans la
variable \textit{thread\_data}.

De même que la fonction précédente si c'est le
premier appel à une fonction de la librairie.



\subsubsection{thread\_yield()}
Cette fonction passe la main à un autre thread si possible. Elle
extrait un thread de \textit{thread\_data} grace à la fonction
\textit{get\_lower\_priority\_thread()}. Un échange de contexte est
ensuite fait pour que le thread courrant ce retrouve dans
\textit{thread\_data} et que le thread retirer soit éxécuté.

Si la valeur de retour de \textit{get\_lower\_priority\_thread()} est
nulle, alors soit \textit{thread\_data} n'a jamais été initialisé
($i.e.$ c'est le premier appel à une fonction de la lib), soit il n'y
a pas d'autre thread en attente d'être éxécuté. Dans les deux cas
aucun changement de context n'est fait.

\subsubsection{thread\_join()}
Cette fonction met en pause le thread courant en attendant la
terminaison du thread correspondant à l'id en paramètre.

Si le thread demandé n'est pas déjà fini alors on se met en attente
passive en fesant pointer son attribut \textit{joiner} vèrs le thread
courant avant de changer de thread et de stocker l'ancien dans
\textit{data}.

Dans le cas où le thread attendu est déjà fini, ou bien après
l'attente, on récupère la valeur de retour si le second argument n'est
pas null, puis on désalloue la structure du thread attendu, grace à la
fonction interne \textit{thread\_delete()}, puisque celui ci est fini.


\subsubsection{thread\_exit()}
Cette fonction interompt le thread courrant.

On fixe la valeur de \textit{isfinished} à $1$, puis on change le
contexte. Pour celà on recherche un nouveau thread à éxécuter avec la
fonction \textit{get\_lower\_priority\_thread} ou on sélectionne un
thread ayant fait un \textit{thread\_join()} sur le thread actuel si
il en existe.

Dans le cas où la valeur de retour de
\textit{get\_lower\_priority\_thread} est nulle et qu'il n'y a pas de
thread ayant fait de join ou bien que c'est le premier appel à une
fonction de la librairie thread alors on quitte le programme
normalement avec un appel à \textit{exit(0)}.


\subsection{TAD de la structure \textit{data}}




\subsection{\textit{mthread}}
Nous avons implémenté une surcouche à notre librairie pour pouvoir
facilement l'interchanger avec la librairie \textit{\textless
 pthread.h\textgreater} de la librairie standard.

Cette surcouche, \textit{mthread\_t}, est implémenté dans
\textit{thread\_mix.c}.  Elle permet de faire appel aux fonctionx
définies précédemant en remplaçant \textit{thread\_*} par
\textit{mthread\_*}. En compilant normalement nos fonctions sont
appelés, avec la macro pré-procéseur \textit{\# define MODE\_PTHREAD} ce
sont les fonctions de la librairie standard qui sont appelées.

Grace à cette surcouche nous avons pu comparer les deux implémentations facilement.



\end{document}
